\newglossaryentry{dlrg}{
	name={DLR},
    description={\dlr{}},
    user1={http://www.dlr.de}
}

\newglossaryentry{fwg}{
	name={FW},
    description={\fw{}},
    user1={http://www.dlr.de/fw/}
}

\newglossaryentry{vfg}{
	name={VF},
    description={\vf{}},
    user1={http://www.dlr.de/vf/}
}

\newglossaryentry{tsg}{
	name={TS},
    description={\ts{}},
    user1={http://www.dlr.de/ts/}
}

\newglossaryentry{fkg}{
	name={FK},
    description={\fk{}},
    user1={http://www.dlr.de/fk/}
}

\newglossaryentry{bmwig}{
	name={BMWi},
    description={\bmwi{}},
    user1={http://www.bmwi.de/}
}

\newglossaryentry{scvssg}{
	name={SC-VSS},
    description={\scvss{}},
    user1={http://www.dlr.de/sc/}
}




\newglossaryentry{software}{
	name={Software},
  	text={software},
    description={El software lo compone el equipamiento lógico de un sistema informático necesarios para la realización de tareas específicas. En Ingeniería de Software se denomina también ``producto''}
}

\newglossaryentry{opensource}{
	name={Open Source}, 
    description={Término con el que se conoce al \gls{software} distribuido y desarrollado libremente. Suele referirse al acceso libre del código fuente del \gls{software}},
    user1={http://opensource.org/}
}

\newglossaryentry{java}{
	name={Java},
    description={Lenguaje de programación orientado a objetos publicado por  Sun Microsystems en 1995 y adquirido por Oracle en 2009},
    user1={https://www.oracle.com/java/}
}

\newglossaryentry{testing}{
	text={testing},
    name={Testing},
    description={Actividad perteneciente al proceso de calidad del \gls{software}. La componen investigaciones empíricas y técnicas de las que se extrae información objetiva e independiente sobre la calidad de un producto \gls{software}}
}

\newglossaryentry{mocking}{
	text={mocking},
    name={Mocking},
    description={Objeto simulado que imita el comportamiento de objetos reales de forma controlada. Utilizado durante el proceso de \gls{testing}, suelen ser usados para pruebas unitarias de \gls{software}}
}

\newglossaryentry{fulltext}{
	name={\fulltext},
    description={Técnica de búsqueda para encontrar en un documento o en un conjunto de ellos examinando todas las palabras almacenadas e intentando que corresponda con los criterios proporcionados. En situaciones donde el número de documentos es alto, la búsqueda Full-Text debe ser precedido de una indexación. Durante la indexación se crea una lista con los términos de los documentos aplicándose la búsqueda sobre ésta de forma más eficiente}}


\newglossaryentry{hardware}{
	name={Hardware},
    text={hardware},
    description={El hardware de un sistema informático lo componen todas sus partes físicas. Sus componentes son eléctricos, electrónicos, electromecánicos y mecánicos}
}


\newglossaryentry{bugT}{
	text={bug tracker},
    name={Bug tracker},
    description={\Gls{software} diseñado para el seguimiento de errores de productos \gls{software}. De esta forma se ayuda a asegurar la calidad de éstos y asistir en el seguimiento de los defectos del producto a las personas involucradas}
}

\newglossaryentry{swIntcon}{
	text={integración contínua},
    name={Integración contínua},
    description={Consiste en la compilación y ejecución de pruebas de forma automática de un proyecto lo más a menudo posible para detectar de esta forma fallos}
}

\newglossaryentry{ideg}{
	text={IDE},
	name={Integrated Development Environment (IDE)},
    description={El entorno de desarrollo integrado es un \gls{software} compuesto por un conjuntos de herramientas de programación. Proveen al desarrollador un marco de trabajo amigable para el proceso de desarrollo }}
    
\newglossaryentry{ui}{
	text={interfaz de usuario},
    name={Interfaz de usuario},
    plural={intefaces de usuario},
    description={Es el medio por el que el usuario puede conectarse con una máquina, equipo o comutadora. Se compone de todos los puntos de contacto del usuario y el equipo. En una intefaz gráfica de \gls{software}, por ejemplo, esta comunicación se produce a través de elementos gráficos}
}

\newglossaryentry{deploy}{
	text={deploying},
    name={Deploying},
    description={Conjunto de actividades realizadas sobre un producto \gls{software} para hacer este disponible a los usuarios}
}


\newglossaryentry{wiremock}{
    name={WireMock},
    description={Biblioteca \gls{opensource} escrita en \gls{java} para simular y \gls{mocking} \glspl{sw} creando un servidor \gls{http} propio},
    user1={http://wiremock.org/}
}

\newglossaryentry{mantis}{
    name={MantisBT},
    description={\Gls{bugT} bajo licencia \gls{opensource} desarrollado en PHP},
    user1={https://www.mantisbt.org/}
}

\newglossaryentry{jenkins}{
    name={Jenkins},
    description={\Gls{software} \gls{opensource} para la integración continua en el servidor},
    user1={http://jenkins-ci.org/}
}

\newglossaryentry{jrebel}{
    name={JRebel},
    description={\Gls{software} para el re\gls{despliegue} automático de cambios en un servidor},
    user1={http://zeroturnaround.com/software/jrebel/}
}

\newglossaryentry{chrome}{
    name={Chrome},
    description={Navegador web ligero implementado por Google},
    user1={http://www.google.com/chrome}
}

\newglossaryentry{chromium}{
    name={Chromium},
    description={Versión \gls{opensource} de \gls{chrome}},
    user1={http://chromium.org/}
}

\newglossaryentry{firefox}{
    name={Mozilla Firefox},
    description={Navegador web de la fundación Mozilla},
    user1={https://www.mozilla.org/es-ES/firefox}
}

\newglossaryentry{ie}{
    name={Internet Explorer},
    description={Navegador web de Microsoft},
    user1={http://windows.microsoft.com/en-us/internet-explorer}
}

\newglossaryentry{safari}{
    name={Safari},
    description={Navegador web desarrollado por Apple},
    user1={https://www.apple.com/safari/}
}

\newglossaryentry{sass}{
    name={Sass},
    description={Extensión de lenguaje de estilos \gls{css}},
    user1={http://sass-lang.com/}
}


\newglossaryentry{logging}{
	text={logging},
    name={Logging},
    description={Grabación secuencial de los acontecimientos que afectan a un producto. Organizados normalmente por orden cronológico, permiten analizar la actividad interna paso a paso y sus interacciones con el medio}
}


\newglossaryentry{framework}{
	text={framework},
   	name={Framework},
    description={Es una estructura conceptual y tecnológica que sirven como base para la organización y desarrollo de \gls{software}}
}

\newglossaryentry{motorbusqueda}{
	text={motor de búsqueda},
    name={Motor de búsqueda},
    description={es un sistema de recuperación de información diseñado para encontrar información almacenada en un sistema informático. Los resultados, \textit{hits}, suelen mostrarse en una lista. Los motores de búsqueda ayudan a minimizar el tiempo necesario para encontrar la información deseada}
}

\newglossaryentry{json}{
	text={JSON},
	name={JavaScript Object Notation (JSON)},
    description={Formato ligero para el intercambio de datos. Creado como subconjunto de la notación literal de objetos en \gls{js}, se ha convertido en una alternativa a \gls{xml} para \gls{ajax}},
    user1={http://json.org/}
}

\newglossaryentry{plugin}{
	text={plugin},
    name={Plugin},
    description={Complemento de una aplicación que añade funcionalidad nueva, generalmente muy específica, a otro producto de \gls{software}}
}

\newglossaryentry{despliegue}{
	text={despliegue},
    name={Despliegue},
    description={},
    see=[Ver:]{deplo}
}

\newglossaryentry{latex}{
	name={\LaTeX},
    sort=L,
    description={Sistema para la composición de textos usado especialmente para la edición de documentos científicos y técnicos},
    user1={http://latex-project.org/}
}

\newglossaryentry{gantt}{
	name={Gantt},
    description={Gráfico de barras, creado por Henry Gantt, que ilustra la planificación temporal de un proyecto}
}

\newglossaryentry{wiki}{
	name={Wiki},
    description={Sitio web colaborativo donde los editores pueden trabajar conjuntamente a través de un navegador web}
}

\newglossaryentry{mensinst}{
	text={mensajería instantánea},
    name={Mensajería instantánea},
    description={También conocido como chat, es una forma de comunicación en tiempo real entre dos o más personas basada en texto}
}

\newglossaryentry{prototipo}{
	text={prototipo},
   	name={Prototipo},
    description={En el ciclo del \gls{software}, ejemplar parcial que intenta simular y muestra algunas propiedades del producto final},
    plural={prototipos}
}

\newglossaryentry{reshfellw}{
	name={Research Fellow},
    description={
		Miembro investigador de una Universidad o Institución. La posición de ``Research Fellow'' requiere normalmente un doctorado, o trabajo equivalente por ejemplo en la industria
    },
    plural=Research Fellows
}

\newglossaryentry{extreprot}{
	name={\textit{Extreme Prototyping}},
    sort=E,
    description={Prototipado extremo \cite{extrPro}}
}

\newglossaryentry{htmlg}{
	text={HTML},
	name={HyperText Markup Language (HTML)},
    description={Lenguaje de marcas de hipertexto para la elaboración de páginas web},
}

\newglossaryentry{bootstrap}{
	text={boostrap},
	name={Boostrap},
    description={\Gls{framework} para frontend creado por Twitter},
    user1={http://getbootstrap.com/}
}
\newglossaryentry{compass}{
	text={Compass},
	name={Compass},
    description={Herramienta \gls{opensource} de ayuda para trabajar con \gls{sass}},
    user1={http://compass-style.org/}
}



\newglossaryentry{nosql}{
	name={NoSQL},
    description={Sistema de almacenaje y acceso a datos modelado por cualquier otro sistema que no sea relaciona. No utiliza como lenguaje SQL como lenguaje principal de consultas},
}

\newglossaryentry{js}{
	name={JavaScript},
    description={Lenguaje de programación interpretado orientado a objetos, basado en prototipos, imperativo, débilmente tipado y dinámico. Se utiliza principalmente en el lado del cliente para mejorar la interfaz de usuario y para la generación de comportamiento dinámico de las páginas webs}
}

\newglossaryentry{ux}{
	name={Experiencia de Usuario},
    text={experiencia de usuario},
    description={Conjunto de factores y elementos relacionados con la interacción del usuario con un sistema obteniendo una percepción positiva o negativa del mismo. Depende de factores relacionados con el diseño y con las emociones percibidas por el usuario},
}

\newglossaryentry{kf}{
	name={\kf},
    sort=K,
    description={Primera versión del framework},
}

\newglossaryentry{kf2}{
	name={\kfII{}},
    sort=K,
    description={Segunda versión del framework},
}

\newglossaryentry{svng}{
	text={SVN},
	name={Subversion (SVN)},
    description={Herramienta \gls{opensource} de control de versiones. Se basa en un repositorio con un funcionamiento similar al de un sistema de ficheros tradicional},
}

\newglossaryentry{umlg}{
	name={Unified Modeling Language (UML)},
    text={UML},
    description={Lenguaje Unificado de Modelado. Lenguaje gráfico para visualizar, especificar, construir y documentar un sistema (\gls{software} o no)},
    user1={http://www.uml.org/}
}


\newglossaryentry{liferay}{
	name={Liferay},
    description={Portal de gestión de contenidos \gls{opensource} escrito en \gls{java}. Incluye un gestor de contenido web permitiendo la construcción de portales y páginas web simplemente conjutando temas, paginas, \glspl{portlet} y una navegación conjunta},
    user1={http://www.liferay.com/}
}


\newglossaryentry{tomcat}{
	name={Tomcat},
    description={Servidor web con soporte de servlets y JSPs escrito en \gls{java}},
    user1={http://tomcat.apache.org/}
}


\newglossaryentry{lucene}{
	name={Lucene},
    description={\Gls{apig} \gls{opensource} para la recuperación de información implementada en \gls{java}. Usado extensamente en la implementación de motores de búsquedas, indexado de documentos y búsquedas \gls{fulltext}},
    user1={http://lucene.apache.org/}
}


\newglossaryentry{solr}{
	name={Solr},
    description={Motor de búsqueda \gls{opensource} implementado en \gls{java} que se ejecuta sobre algún contenedor de servlets \gls{java}. Basado en \gls{lucene}, ofrece \glspl{api} en \gls{xml} y \gls{json}, resaltado de texto en resultados, búsqueda por facetas, caché y una interfaz de administración},
    user1={http://lucene.apache.org/solr}
}

\newglossaryentry{jetty}{
	name={Jetty},
    description={Servidor opensource de HTTP implementado en Java y contenedor de servlets},
    user1={http://jetty.mortbay.org/jetty/index.html}
}


\newglossaryentry{maven}{
	name={\maven{}},
    description={Herramienta de \gls{software} para la gestión y construcción de proyectos en Java. Su característica clave es la descarga dinámica desde su repositorio de proyectos \gls{opensource} en diferentes versiones},
    user1={http://maven.apache.org/}
}



\newglossaryentry{jsdkg}{
	name={Java SE Development Kit},
    text={Java SDK},
    description={\Gls{software} que provee herramientas para el desarrollo de aplicaciones Java},
}

\newglossaryentry{ajaxg}{
	text={AJAX},
	name={Asynchronous JavaScript and XML (AJAX)}, 
    description={Conjunto de técnicas de desarrollo basado en los estándares actuales. Proporciona una forma de obtener datos desde el servidor, actualizar partes de una página web sin tener que recargar la página completamente}
}

\newglossaryentry{riag}{
	text={RIA}, 
    name={Rich Internet Application (RIA)}, 
    description={Aplicaciones de Internet enriquecidas, aplicaciones web ejecutadas en un navegador web con las características aplicaciones de escritorio 
    tradicionales. Surgen como una combinación de las ventajas que ofrecen las aplicaciones web y las aplicaciones tradicionales para mejorar la experiencia y productividad del usuario},
}    



\newglossaryentry{gwtg}{
	name={Google Web Toolkit (GWT)},
    text={GWT},
    description={Framework de Google para crear a partir de código Java elementos y código \gls{html} y \gls{js} compatible cada naegador},
    user1={http://www.gwtproject.org/}
}

\newglossaryentry{xmlg}{
	text={XML},
	name={Extensible Markup Language (XML)},
    description={Lenguaje de marcas extensible desarrollado para almacenar datos en forma legible. Permite definir gramáticas especificas de lenguaje para estructurar documentos de gran dimensión},
    user1={http://www.w3.org/TR/REC-xml/}
}


\newglossaryentry{urlg}{
	text={URL},
	name={Uniform Resource Locator (URL)},
    description={Cadena de caracteres con la que se asigna una dirección única de Internet a cada recurso de información disponible}
}

\newglossaryentry{portlet}{
	name={Portlet},
    text={portlet},
    description={Componentes de la interfaz de usuario gestionadas y visualizadas en un portal web},
}

\newglossaryentry{html5}{
	name={HTML5},
    description={Quinta versión del lenguaje de marcado \gls{html}. Sus principales mejoras son el tratamiento de objetos multimedia y elementos de interacción gráfica},
}

\newglossaryentry{cron}{
	name={Cron},
    description={Administrador de procesos en segundo plano de sistemas UNIX. Permite ejecutar estos procesos en intervalos regulares de tiempo},
}

\newglossaryentry{monitor}{
	name={\mo{}},
    sort=M,
    description={},
    user1={http://monitorportal.dlr.de/},  
}

\newglossaryentry{cs}{
	name={\cs{}},
    sort=S,
    description={},
    user1={http://daten.clearingstelle-verkehr.de/}
}

\newglossaryentry{elib}{
	name={ELIB-Portal},
    description={},
    user1={http://elib.dlr.de/}
}

\newglossaryentry{strada}{
	name={\stradai{}}, 
	description={Search TRAsport DAta},
    sort=S
}

\newglossaryentry{eclipse}{
	name={Eclipse}, 
    description={\Gls{software} \gls{opensource} multiplataforma escrita el Java para el desarrollo de aplicaciones. Aunque en sus orígenes sólo aceptaba Java como lenguaje de programación, gracias a sus gran número de \glspl{plugin}, en la actualidad puede considerarse como la interfaz de programación más versátil},
}


\newglossaryentry{scroll}{
	name={Scroll},
    text={scroll},
    description={Desplazamiento en 2D de los contenidos que conforman elemento de la interfaz de usuario, por ejemplo una lista o una ventana del navegador web},
}


\newglossaryentry{paginacion}{
	name={Paginación},
    text={paginación},
    description={División de una lista o documento en diferentes partes llamadas páginas y la numeración de las mismas },
}

\newglossaryentry{flash}{
	name={Flash},
    description={\Gls{software} propietario utilizado tradicionalmente para la generación de animaciones. Estas pueden ser visualizadas en un navegador web o a través de un reproductor de Flash. Actualmente su utilización ha decaído considerablemente por sus fallos de seguridad y la aparición de nuevas tecnologías como \gls{html5}},
}

\newglossaryentry{metadato}{
	name={(Meta-)dato},
    sort=M,
    description={Se consideran datos que ayudan a describir a otros datos. De esta forma, a través de los (Meta-)datos se facilita la labor de clasificar y encontrar los datos que son descritos por ellos},
}

\newglossaryentry{sigma}{
	name={Sigma.js}, 
    description={Biblioteca gráfica de \gls{js} para dibujar gráficos},
    user1={http://sigmajs.org/}
}

\newglossaryentry{cyto}{
	name={Cytoscape}, 
    description={Plataforma para la visualización y análisis de redes complejas},
    user1={http://www.cytoscape.org/}
}
\newglossaryentry{d3}{
	name={D3.js (Data-Driven Documents)},
    text={D3.js},
    description={Biblioteca \gls{js} para la manipulación de documentos basándose en datos},
    user1={http://d3js.org/}
}

\newglossaryentry{canvas}{
	name={Canvas}, 
    description={Elemento de \gls{html5} que permite la generación dinámica de gráficos a través de la ejecución de código. Estos gráficos pueden ser estáticos y con animaciones}
}

\newglossaryentry{webgl}{
	name={WebGL}, 
    description={\Gls{api} para \gls{js} que permite usar la implementación nativa de OpenGL ES 2.0 que en un futuro será incorporada a los navegadores web. Actualmente se utiliza para la visualización de gráficos en 3D en la web},
    user1={http://www.khronos.org/webgl/}
}

\newglossaryentry{svg}{
	name={SVG}, 
    description={Gráficos Vectoriales Redimensionables tanto estáticos como animados en formato XML},
    user1={http://www.w3.org/Graphics/SVG/}
}

\newglossaryentry{css}{
	name={CSS}, 
    description={Hoja de estilo en cascada es usado como lenguaje para definir la presentación de un documento estructurado en \gls{html}}
}
\newglossaryentry{leaflet}{
	name={Leaflet}, 
    description={Biblioteca \gls{opensource} escrita en \gls{js} para la interacción con mapas},
    user1={http://leafletjs.com/}
}

\newglossaryentry{crossfilter}{
	name={Crossfilter}, 
    description={Biblioteca \gls{opensource} escrita en \gls{js} para explorar conjuntos de datos multivariables de gran tamaño en un navegador web.},
    user1={http://square.github.io/crossfilter/}
}

\newglossaryentry{solrj}{
	name={Solrj}, 
    description={Cliente \gls{java} para acceder a \gls{solr}. Ofrece una interfaz para añadir, actualizar y consultar índices de \gls{solr}},
    user1={http://wiki.apache.org/solr/Solrj}
}
\newglossaryentry{firewall}{
	name={Firewall},
    text={firewall},
    description={Parte de un sistema o red diseñado para filtrar el acceso. Por lo tanto, permite para bloquear el acceso no autorizado y permitir las comunicaciones autorizadas}
}

\newglossaryentry{sw}{
	name={Servicio web},
    text={servicio web},
    plural={servicios web},
    description={Tecnología que utiliza un conjunto de protocolos y estándares relacionados con la web para intercambiar datos entre aplicaciones \gls{software}}
}

\newglossaryentry{vaadin}{
	name={Vaadin}, 
    description={\Gls{framework} de programación \gls{opensource} para aplicaciones \gls{ria}  programado en \gls{java}. Su funcionamiento se basa en una arquitectura del lado del servidor. Esto significa que la mayoría de la lógica transcurre este servidor. Usando \gls{ajax}, se produce la comunicación del navegador web con el servidor asegurando de esta forma la interacción del usuario con los componentes. En la parte superior del cliente se encuentran los componentes de Vaadin los cuales pueden ser ampliados usando \gls{gwt}},
    user1={https://vaadin.com/}
}

%%% The glossary entry the acronym links to   
\newglossaryentry{apig}{
	text={API},
	name={Application Programming Interface (API)},
    description={Interfaz de programación de aplicaciones, conjunto de reglas y especificaciones que pueden ser usados por un programa para tener acceso y hacer uso de los servicios y recursos proporcionados por otro el cual implementa esta API}
}

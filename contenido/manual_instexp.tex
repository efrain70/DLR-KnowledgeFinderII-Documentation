% ------------------------------------------------------------------------------
% Este fichero es parte de la plantilla LaTeX para la realización de Proyectos
% Final de Grado, protegido bajo los términos de la licencia GFDL.
% Para más información, la licencia completa viene incluida en el
% fichero fdl-1.3.tex

% Copyright (C) 2012 SPI-FM. Universidad de Cádiz
% ------------------------------------------------------------------------------

Las instrucciones de uso del \gls{framework} \gls{kf2} aplicado al \gls{strada} se detallan a continuación.

\section{\IfLanguageName{english}{Introduction}{Introducción}}
% Resumen de los principales objetivos, ámbito y alcance del software desarrollado.

En el presente capitulo se desarrollará el manual de instalación y explotación del índice de búsqueda en \gls{solr}, del servicio web \glslink{despliegue}{desplegado} en \gls{liferay} y de la \gls{ui} en un \gls{portlet} del proyecto \gls{strada} implementado con el \gls{framework} \gls{kf2}.\\

Esta aplicación web permite consultar las bases de datos de ``\transmov'' de los institutos del \glsfirst{dlr} \glsfirst{fw} y \glsfirst{vf} conjuntamente los documentos de ambos portales ofreciendo un sistema de búsqueda avanzada y optimizada sobre los \glspl{metadato} de los documentos de los institutos involucrados.\\

El portal ofrece una visualización del conocimiento interactiva de los \glspl{metadato}  que representa las relaciones espaciales, temporales y contextuales entre los documentos y fuentes de datos. Esta representación aporta nuevos conocimientos que facilitan las tareas de investigación.\\




Este manual se divide en tres módulos que se tratarán de expondrán de forma independiente:
\begin{itemize}
	\item Índice \gls{solr}
	\item \Gls{sw} en \gls{liferay}
	\item \Gls{portlet} de visualización
\end{itemize}

\section{Requisitos generales previos}
A continuación se exponen los requisitos necesarios comunes a todos los módulos que se describirán en este manual.

\subsection{Java}
Basándonos en la documentación de \gls{solr} 4.8.1 \cite{solrinstall}, el servidor requiere la versión de \gls{jsdk} 7 o superior. Aunque impuesto por \gls{solr}, se recomienda usar la misma versión de \gls{jsdk} para los restantes módulos.\\

Una vez se tenga instalado, hay que configurar la variable del sistema $JAVA\_HOME$ para usar el \gls{jsdk} que acabamos de instalar.

\subsection{Eclipse}
\label{subsubsection:eclipsesolr}
Para el desarrollo de este software se recomienda utilizar el \gls{ide} \gls{eclipse}. Éste se puede descargar en la dirección oficial \url{http://www.eclipse.org/downloads/}.

\subsection{\maven{} 3}
La instalación de \gls{maven} se puede realizar en algunos sistemas operativos, como Ubuntu, desde los repositorios oficiales. Desde la dirección oficial (\url{http://maven.apache.org/download.html}) también se pueden descargar e instalar fácilmente.\\

Si no se ha configurado automáticamente durante el paso anterior, hay que crear una variable de sistema $M2\_HOME$ que apunte al directorio donde se ha instalado, por ejemplo:
\mbox{C:\textbackslash{}Program Files(x86)\textbackslash{}apache-maven-3.1.0}.\\

Si la organización cuenta con un repositorio de \gls{maven} propio, éste se puede añadir editando el fichero \textit{settings.xml} de \gls{maven}. Por ejemplo, para el \gls{scvss}, el repositorio apunta a \url{http://repo.sistec.dlr.de/nexus/content/groups/public/}.\\

Para usarlo conjuntamente con \gls{eclipse} (apartado \ref{subsubsection:eclipsesolr}) se recomienda instalar el \gls{plugin} Maven2Eclipse desde la dirección \url{http://download.eclipse.org/technology/m2e/releases}

\subsection{Cliente \Gls{svn}}
Para poder obtener el código y trabajar con el repositorio \gls{svn} es necesario instalar un cliente de \gls{svn} 1.7 o superior.\\

También se puede optar por instalar directamente un \gls{plugin} para \gls{eclipse} que incluya este cliente, por ejemplo Subclipse que está disponible bajo la dirección \url{http://subclipse.tigris.org/update_1.10.x}.

\subsection{\Gls{mantis} y \gls{jenkins}}
\Gls{eclipse} se conecta con \gls{mantis} y \gls{jenkins} a través de Mylyn. Este \gls{plugin} está disponible en la dirección \url{http://download.eclipse.org/mylyn/releases/latest}.\\

Para configurar en Mylyn \gls{mantis} en \gls{eclipse} siga los siguientes pasos:
\begin{enumerate} 
	\item Window > Show View > Task List
	\item Click on Add MyLyn Support
	\item Install connector (Mantis)
	\item Boton derecho en task view, add query
	\item Add new repository
		\subitem Type: Mantis
		\subitem Path: https://www.sistec.dlr.de/mantis
		\subitem User/Pass: Tus credenciales
		\subitem Validate
	\item Crear las consultas (queries) deseadas.
\end{enumerate}

A continuación, la configuración de Mylyn con \gls{jenkins}:
\begin{enumerate}
	\item Window > Show View > Builds
	\item Add build
		\subitem Type: Hudson [Jenkins]
		\subitem Path: https://ci.sc.dlr.de/jenkins
		\subitem User/Pass: Tus credentials
		\subitem Validate
	\item Elegir los builds de \gls{kf2} de la lista.
\end{enumerate}

\section{Índice Solr}
\label{section:manualsolr}
En este apartado se describe el manual de desarrollo, instalación y explotación del sistema de importación de datos y su aplicación en un servidor \gls{solr}.

\subsection{\IfLanguageName{english}{Requirements}{Requisitos previos}}

\subsubsection{Apache \gls{solr} 4.8.1}
\label{subsubsection:solrinstall}
Desde la dirección \url{http://www.apache.org/dyn/closer.cgi/lucene/solr/4.8.1} nos podemos descargar \gls{solr}. Una vez descargado y descomprimido en un directorio, por ejemplo ``UnzipDirectory'', ejecutamos el servidor \gls{jetty} de ejemplo (código \ref{code:jettysorl}).\\

\begin{listing}[H]
    \begin{minted}[linenos,
               numbersep=5pt,
               frame=single,
               framesep=2mm]{bash}

java -jar [UnzipDirectory]/solr-4.8.1/example/start.jar
    \end{minted}
    \caption{Ejemplo de ejecución inicial de \gls{solr} en \gls{jetty}}
    \label{code:jettysorl}
\end{listing}


Si todo ha funcionado correctamente, \gls{solr} estaría accesible en la dirección \url{http://localhost:8983/solr}.


%\section{\IfLanguageName{english}{}{Inventario de componentes}}
%Lista de los componentes hardware y software que se incluyen en la versión del
% producto.

\subsection{\IfLanguageName{english}{Installation and Setup
procedures}{Procedimientos de instalación}}
%Procedimientos de instalación y configuración de cada componente hardware y software (base y desarrollado) para asegurar la correcta instalación y explotación del sistema, así como aquellos procedimientos necesarios de migración/carga de datos.

\subsubsection{Obtención del \gls{software}}
Checkout desde \gls{svn} el índice desde \url{https://svn.sistec.dlr.de/svn/xps/xps-server/branches/visual/solrIndex/} si no se dispone de una copia de trabajo.\\

Checkout desde \gls{svn} del sistema de importación desde \url{https://svn.sistec.dlr.de/svn/xps/xps-server/branches/visual/de.dlr.xps.server.knowledgefinder.handler.dataimport/} si no se dispone de una copia de trabajo.\\

\subsubsection{Construcción del \gls{software}}
Situándonos en el directorio donde se ha realizado el checkout, ejecutamos las lineas expuestas en el código \ref{code:solrbuild1} y \ref{code:solrbuild2}.

\begin{listing}[H]
    \begin{minted}[linenos,
               numbersep=5pt,
               frame=single,
               framesep=2mm]{bash}

cd ./de.dlr.xps.server.knowledgefinder.handler.dataimport/
mvn clean install
    \end{minted}
    \caption{Construcción con \gls{maven} de DataImport}
    \label{code:solrbuild1}
\end{listing}

\begin{listing}[H]
    \begin{minted}[linenos,
               numbersep=5pt,
               frame=single,
               framesep=2mm]{bash}

cd ../solrIndex/
mvn clean install
    \end{minted}
    \caption{Construcción con \gls{maven} de índice de búsqueda}
    \label{code:solrbuild2}
\end{listing}


A través de \gls{eclipse} se puede realizar también siguiendo estos pasos:
\begin{itemize}
	\item Añadir un nuevo ``Maven Run Configuration'' usando como directorio base\\
          $\$\{workspace\_loc:\/de.dlr.xps.server.knowledgefinder.handler.dataimport\}$\\
          y como ``goal'' clean install para construir el sistema de importación.
	\item Añadir un nuevo ``Maven Run Configuration'' usando como directorio base\\ 			  $\$\{workspace\_loc:\/solr\_index\}$\\
		  y como ``goal'' clean install para construir el índice.
\end{itemize}

\subsubsection{\Gls{despliegue} en el \gls{solr}}
Para concluir se debe de configurar \gls{solr} para que utilice el índice. Para ello se debe ejecutar el servidor \gls{jetty} como se vio en el apartado \ref{subsubsection:solrinstall} pero indicando la variable $solr.solr.home$ (código \ref{code:solrdeploy}).

\begin{listing}[H]
    \begin{minted}[linenos,
               numbersep=5pt,
               frame=single,
               framesep=2mm]{bash}

java -jar -Dsolr.solr.home={workspace_loc}/solr_index
[UnzipDirectory]/solr-4.8.1/example/start.jar
    \end{minted}
    \caption{Ejemplo de ejecución de \gls{solr} en \gls{jetty} con el índice de \gls{strada}}
    \label{code:solrdeploy}
\end{listing}

\subsubsection{Configuración del índice}
Después de la instalación, hay que configurar el origen de los datos.\\

Los datos se pueden obtener desde el repositorio \gls{svn} (\url{https://svn-dmz.sistec.dlr.de/svn/xps/fw-test-data/}) y configurar el índice con la ruta de la copia de trabajo local (código \ref{code:importlocal}).

\begin{listing}[H]
    \begin{minted}[linenos,
               numbersep=5pt,
               frame=single,
               framesep=2mm]{xml}

...
<entity name="strada"
processor="SVNEntityProcessor"
transformer="..."
query="/KFdata/dat/CS"
...
>
...
<entity name="fw"
processor="SVNEntityProcessor"
transformer="..."
query="/KFdata/dat/FW"
...
>
...
    \end{minted}
    \caption{Configuración del índice desde copia de trabajo local de \gls{svn}}
    \label{code:importlocal}
\end{listing}


Otra opción es que la importación se realice directamente leyendo el repositorio de \gls{svn} remoto. Para ello hay que configurar también las credenciales para acceder al mismo (código \ref{code:importremote})\\

\begin{listing}[H]
    \begin{minted}[linenos,
               numbersep=5pt,
               frame=single,
               framesep=2mm]{xml}

...
<entity name="strada"
processor="SVNEntityProcessor"
transformer="..."
query="https://svn-dmz.sistec.dlr.de/svn/xps/fw-test-data/dat/CS"
username="usernameSVN"
password="passwordSVN"
...
>
...
<entity name="fw"
processor="SVNEntityProcessor"
transformer="..."
query="https://svn-dmz.sistec.dlr.de/svn/xps/fw-test-data/dat/FW"
username="usernameSVN"
password="passwordSVN"
...
>
...
    \end{minted}
    \caption{Configuración del índice desde repositorio remoto de \gls{svn}}
    \label{code:importremote}
\end{listing}


\subsubsection*{Configuración en servidor}
En el código \ref{code:solrdebian} se muestra un ejemplo de configuración para servicio en una máquina de la familia Debian.
\begin{listing}[H]
    \begin{minted}[linenos,
               numbersep=5pt,
               frame=single,
               framesep=2mm]{bash}

#!/bin/bash
#startup script for the Solr server
#
# description: Starts and stops the Solr with Jetty daemon.
# processname: solr (java)
# pidfile: /var/run/solr.pid
JAVA_7_HOME=/usr/lib/jvm/jdk1.7.0_60/
SOLR_HOME=/home/lima_ef/solrIndex/
JETTY_HOME=/opt/solr-4.8.1/example/
STOP_PORT=8079
STOP_KEY=keyStopSolr
case $1
in
start)
export JAVA.HOME=$JAVA_7_HOME
java -jar -Dsolr.solr.home=$SOLR_HOME -DSTOP.PORT=$STOP_PORT 
          -Djetty.home=$JETTY_HOME
/opt/solr-4.8.1/example/start.jar --start
;;
stop)
export JAVA.HOME=$JAVA_7_HOME
java -jar -Dsolr.solr.home=$SOLR_HOME -DSTOP.PORT=$STOP_PORT 
          -Djetty.home=$JETTY_HOME
/opt/solr-4.8.1/example/start.jar --stop
;;
restart)
export JAVA.HOME=$JAVA_7_HOME
java -jar -Dsolr.solr.home=$SOLR_HOME -DSTOP.PORT=$STOP_PORT 
          -Djetty.home=$JETTY_HOME
/opt/solr-4.8.1/example/start.jar --stop
java -jar -Dsolr.solr.home=$SOLR_HOME -DSTOP.PORT=$STOP_PORT 
          -Djetty.home=$JETTY_HOME
/opt/solr-4.8.1/example/start.jar --start
;;
*)
echo "Usage: /etc/init.d/solr start|stop|restart"
;;
esac
exit 0
    \end{minted}
    \caption{Ejemplo de servicio para \gls{solr} en una máquina de Debian (\/etc\/init.de\/solr}
    \label{code:solrdebian}
\end{listing}

\subsection{\IfLanguageName{english}{Setup Testing}{Pruebas de implantación}}
% Descripción de las pruebas a realizar después de la instalación del sistema. 
Para comprobar que todo el proceso se ha realizado correctamente, la importación debe ser exitosa. Para ello:
\begin{enumerate}
	\item Comprobar que en la dirección \url{http://localhost:8983/solr/#/strada} se encuentran las propiedades generales del índice.
	\item Recargar el índice:
		\subitem Ir a \textit{Core Admin}.
		\subitem Seleccionar el índice para \gls{strada}.
		\subitem Pulsar \textit{Reload}.
	\item Ir a la dirección \url{http://localhost:8983/solr/#/strada/dataimport//dataimport}
	\item Pulsar \textit{Execute}
	\item Para ver si los datos se han importado correctamente, pulsar \textit{Refresh Status} y comprobar el número de documentos importados.
	\item Mediante una simple consulta, se puede ver cómo han sido importados los datos. Por ejemplo, visitando \url{http://localhost:8983/solr/strada/select?q=*\%3A*&wt=json&indent=true} se obtienen todos los datos importados en formato \gls{json}.
\end{enumerate}

\subsection{\IfLanguageName{english}{Operating and Service-level
Procedures}{Procedimientos de operación y nivel de servicio}}
% Procedimientos necesarios para asegurar el correcto funcionamiento, rendimiento, disponibilidad y seguridad del sistema: back-ups, chequeo de logs, etc. También, es preciso indicar claramente aquellas actuaciones precisas necesarias para el mantenimiento preventivo del sistema y así prevenir posibles fallos en el mismo.
\Gls{solr} dispone de su propio sistema de \gls{logging}. Durante toda la ejecución de \gls{solr} se puede consultar a través de la interfaz de administración o en la dirección \url{http://localhost:8983/solr/admin/info/logging} (formato \gls{xml}).

\section{\Gls{sw} en \gls{liferay}}
\label{section:manualsw}
En esta sección se describe el manual de desarrollo, instalación y explotación del \gls{sw} que se aloja en \gls{liferay} para la utilización del indice \gls{solr} anteriormente 
explicado.

\subsection{\IfLanguageName{english}{Requirements}{Requisitos previos}}

\subsubsection{Índice \gls{solr}}
Como es de esperar, este \gls{sw} necesita una instancia del índice \gls{solr} para el proyecto \gls{strada} en funcionamiento. En el punto anterior \ref{section:manualsolr} se explica cómo realizarlo.

\subsubsection{\Gls{liferay} v6.2 CE Server (\gls{tomcat} 7)}
\label{subsubsection:reqliferay}
Desde la dirección \url{http://sourceforge.net/projects/lportal/files/Liferay\%20Portal/6.2.1\%20GA2/liferay-portal-tomcat-6.2-ce-ga2-20140319114139101.zip/download} nos podemos descargar \gls{liferay}. Una vez descargado, lo descomprimimos en un directorio, por ejemplo ``UnzipDirectory''.\\

Para ejecutar el servidor \gls{tomcat} con \gls{liferay} de ejemplo (código \ref{code:liferaystart}).\\
\begin{listing}[H]
    \begin{minted}[linenos,
               numbersep=5pt,
               frame=single,
               framesep=2mm]{bash}
               
java -jar [UnzipDirectory]/liferay-portal-6.2-ce-ga2/tomcat-7.0.42/
		  bin/catalina.sh start
    \end{minted}
    \caption{Ejemplo de ejecución inicial de \gls{liferay} en \gls{tomcat}}
    \label{code:liferaystart}
\end{listing}

Si todo ha funcionado correctamente, \gls{liferay} estaá accesible en la dirección \url{http://localhost:8080/} donde se mostrará el formulario para la configuración del portal.\\

Para trabajar más cómodamente en \gls{eclipse} con \gls{maven} y \gls{liferay}, se recomienda instalar los \glspl{plugin} Liferay IDE y m2e-liferay disponibles en \url{http://releases.liferay.com/tools/ide/latest/milestone/}.

\subsection{\IfLanguageName{english}{Installation and Setup
procedures}{Procedimientos de instalación}}
%Procedimientos de instalación y configuración de cada componente hardware y software (base y desarrollado) para asegurar la correcta instalación y explotación del sistema, así como aquellos procedimientos necesarios de migración/carga de datos.

\subsubsection{Obtención del \gls{software}}
Realizamos un checkout del código del \gls{sw} desde \gls{svn}  \url{https://svn.sistec.dlr.de/svn/xps/xps-server/branches/visual/webservice/} si no se dispone de una copia de trabajo.

\subsubsection{Construcción del \gls{software}}
Situándonos en el directorio donde se ha realizado el checkout, ejecutamos las lineas expuestas en el código \ref{code:swbuild}.

\begin{listing}[H]
    \begin{minted}[linenos,
               numbersep=5pt,
               frame=single,
               framesep=2mm]{bash}
               
cd ./webservice/
mvn clean install
    \end{minted}
    \caption{Construcción con \gls{maven} del \gls{sw}}
    \label{code:swbuild}
\end{listing}

A través de \gls{eclipse} se puede realizar también añadiendo un nuevo ``Maven Run Configuration'' usando como directorio base \\
$\$\{workspace\_loc:\/webservice\}$\\
y como ``goal'' clean install para construir el \gls{sw}.

\subsubsection{\Gls{despliegue} en \gls{liferay}}
Gracias al \gls{plugin} anteriormente indicado, en el archivo de configuración de \gls{maven} del \gls{sw} se puede configurar para que \gls{eclipse} realice los \glspl{despliegue} en el portal \gls{liferay} de forma automática. Para ello hay que editar el fichero $pom.xml$ como se indica en el código \ref{code:pomsw}.

\begin{listing}[H]
    \begin{minted}[linenos,
               numbersep=5pt,
               frame=single,
               framesep=2mm]{xml}
               
<properties>
...
<liferay.version>6.2.1</liferay.version>
<liferay.auto.deploy.dir>
	[UnzipDirectory]/liferay-portal-6.2-ce-ga2/deploy/
</liferay.auto.deploy.dir>
<liferay.maven.plugin.version>
	6.2.1
</liferay.maven.plugin.version>
...
</properties>
...
    \end{minted}
    \caption{Extacto del fichero de configuración de \gls{maven} para auto-\gls{despliegue} con \gls{liferay}}
    \label{code:pomsw}
\end{listing}


y luego ejecutar el \gls{plugin} desde la consola (código \ref{code:liferayautocons}) o configurando \gls{eclipse} añadiendo un nuevo ``Maven Run Configuration'' usando como directorio base\\
$\$\{workspace\_loc:\/webservice\}$\\
y como ``goal'' liferay:deploy.

\begin{listing}[H]
    \begin{minted}[linenos,
               numbersep=5pt,
               frame=single,
               framesep=2mm]{bash}
               
cd ./webservice/
mvn liferay:deploy
    \end{minted}
    \caption{Auto-\gls{deploy} con \gls{maven} del \gls{sw}}
    \label{code:liferayautocons}
\end{listing}


\subsubsection{Configuración del \gls{sw}}

También se debe de configurar el \gls{sw} para que utilice el índice \gls{solr}. Para ello se debe adaptar el fichero $webservice.properties$ (código \ref{code:swpropert}).

\begin{listing}[H]
    \begin{minted}[linenos,
               numbersep=5pt,
               frame=single,
               framesep=2mm]{bash}
               
##
## Solr connection
##
solr.scheme=http
solr.host=localhost
solr.port=8983
# solr.username=username
# solr.password =password
solr.core=solr/strada
    \end{minted}
    \caption{Ejemplo de configuración del \gls{sw} para usar el índice de \gls{solr}}
    \label{code:swpropert}
\end{listing}

\subsubsection*{Configuración en servidor}
En el código \ref{code:liferaydebian} se muestra un ejemplo de configuración para crear un servicio en una máquina de la familia Debian.

\begin{listing}[H]
    \begin{minted}[linenos,
               numbersep=5pt,
               frame=single,
               framesep=2mm]{bash}

#!/bin/bash
#
#
#
#
#
Startup script for the Tomcat server
description: Starts and stops the Tomcat daemon.
processname: liferay (java)
pidfile: /var/run/liferay.pid
# See how we were called.
JAVA_7_HOME="/usr/lib/jvm/jdk1.7.0_60/"
case $1
in
start)
export JAVA_HOME=$JAVA_7_HOME
su - liferay -c 'sh /opt/liferay-portal-6.2-ce-ga2/tomcat-7.0.42/bin/startup.sh'
;;
stop)
export JAVA_HOME=$JAVA_7_HOME
su - liferay -c 'sh /opt/liferay-portal-6.2-ce-ga2/tomcat-7.0.42/bin/shutdown.sh'
;;
restart)
export JAVA_HOME=$JAVA_7_HOME
su - liferay -c 'sh /opt/liferay-portal-6.2-ce-ga2/tomcat-7.0.42/bin/shutdown.sh'
su - liferay -c 'sh /opt/liferay-portal-6.2-ce-ga2/tomcat-7.0.42/bin/startup.sh'
;;
*)
echo "Usage: /etc/init.d/liferay start|stop|restart"
;;
esac
exit 0
    \end{minted}
    \caption{Ejemplo de servicio para \gls{liferay} en una máquina de Debian}
    \label{code:liferaydebian}
\end{listing}


\subsection{\IfLanguageName{english}{Setup Testing}{Pruebas de implantación}}
% Descripción de las pruebas a realizar después de la instalación del sistema. 
Para comprobar que todo el proceso se ha realizado correctamente, simplemente hay que ejecutar algunas de las funciones del servidor. Utilizando la interfaz para \glspl{sw} de \gls{liferay} se pueden usar los formularios que automáticamente éste genera y ver que las peticiones realizan su cometido:
\begin{enumerate}
	\item Abrir la dirección \url{http://localhost:8080/api/jsonws}.
    \item Seleccionar la \gls{api} de \gls{kf2}.
    \item Utilizar el formulario para ejecutar alguna funcion de la \gls{api}.
    \item Comprobar el resultado abriendo la \gls{url} generada.
\end{enumerate}

\subsection{\IfLanguageName{english}{Operating and Service-level
Procedures}{Procedimientos de operación y nivel de servicio}}
% Procedimientos necesarios para asegurar el correcto funcionamiento, rendimiento, disponibilidad y seguridad del sistema: back-ups, chequeo de logs, etc. También, es preciso indicar claramente aquellas actuaciones precisas necesarias para el mantenimiento preventivo del sistema y así prevenir posibles fallos en el mismo.
Cada petición de \gls{liferay} queda registrado en un sistema de \gls{logging} propio de \gls{tomcat}. Durante toda la ejecución del \gls{sw} se puede consultar a través de los ficheros de la carpeta [UnzipDirectory]/liferay-portal-6.2-ce-ga2/tomcat-7.0.42/logs/.

\section{Visualización}
En esta sección se describe el manual de desarrollo, instalación y explotación del \gls{portlet} alojado en \gls{liferay} encargado de la visualización.

\subsection{\IfLanguageName{english}{Requirements}{Requisitos previos}}

\subsubsection{\Gls{sw} accesible}
La visualización requiere del \gls{sw} explicado en la sección anterior (\ref{section:manualsw}. Necesita una instancia del índice \gls{sw} para el proyecto \gls{strada} en funcionamiento. 

\subsubsection{\Gls{liferay} v6.2 CE Server (\gls{tomcat} 7)}
Ver apartado \ref{subsubsection:reqliferay}

\subsection{\IfLanguageName{english}{Installation and Setup
procedures}{Procedimientos de instalación}}
%Procedimientos de instalación y configuración de cada componente hardware y software (base y desarrollado) para asegurar la correcta instalación y explotación del sistema, así como aquellos procedimientos necesarios de migración/carga de datos.

\subsubsection{Obtención del \gls{software}}
Obtener el código del componente de visualización desde \gls{svn} (\url{https://svn.sistec.dlr.de/svn/xps/xps-server/branches/visual/de.dlr.xps.server.knowledgefinder.frontend/}) si no se dispone de una copia de trabajo.

\subsubsection{Construcción del \gls{software}}
Situándonos en el directorio donde se ha realizado el checkout, ejecutamos las lineas expuestas en el código \ref{code:visbuild}.


\begin{listing}[H]
    \begin{minted}[linenos,
               numbersep=5pt,
               frame=single,
               framesep=2mm]{bash}

cd ./de.dlr.xps.server.knowledgefinder.frontend/
mvn clean install
    \end{minted}
    \caption{Construcción con \gls{maven} de la visualización}
    \label{code:visbuild}
\end{listing}

A través de \gls{eclipse} se puede realizar también añadiendo un nuevo ``Maven Run Configuration'' usando como directorio base\\ 
$\$\{workspace\_loc:\/de.dlr.xps.server.knowledgefinder.frontend\}$\\
y como ``goal'' clean install para construir el \gls{portlet} para la visualización.

\subsubsection{\Gls{despliegue} en \gls{liferay}}
Gracias al \gls{plugin} anteriormente indicado, en el archivo de configuración de \gls{maven} de la visualización se puede configurar para que \gls{eclipse} realice los \glspl{despliegue}
 en el portal \gls{liferay} de forma automática. Para ello hay que editar el fichero $pom.xml$ como se indica en el código \ref{code:pomvisual}.

\begin{listing}[H]
    \begin{minted}[linenos,
               numbersep=5pt,
               frame=single,
               framesep=2mm]{xml}
<properties>
...
<liferay.version>6.2.1</liferay.version>
<liferay.auto.deploy.dir>
	[UnzipDirectory]/liferay-portal-6.2-ce-ga2/deploy/
</liferay.auto.deploy.dir>
<liferay.maven.plugin.version>6.2.1</liferay.maven.plugin.version>
...
</properties>
...
    \end{minted}
    \caption{Extracto del fichero de configuración de \gls{maven} para auto-\gls{deploy} con \gls{liferay}}
    \label{code:pomvisual}
\end{listing}


y luego ejecutar el \gls{plugin} desde la cosola (código \ref{code:liferayautoconsfront}) o configurando \gls{eclipse} añadiendo un nuevo ``Maven Run Configuration'' usando como directorio base\\ 
$\$\{workspace\_loc:\/de.dlr.xps.server.knowledgefinder.frontend\}$\\
y como ``goal'' liferay:deploy.\\

\begin{listing}[H]
    \begin{minted}[linenos,
               numbersep=5pt,
               frame=single,
               framesep=2mm]{bash}
               
cd ./de.dlr.xps.server.knowledgefinder.frontend/
mvn liferay:deploy
    \end{minted}
    \caption{Auto-\gls{deploy} con \gls{maven} del \gls{sw}}
    \label{code:liferayautoconsfront}
\end{listing}


Una vez que el \gls{portlet} ha sido \glslink{despliegue}{desplegado}, hay que añadirlo a una página de \gls{liferay} siguiendo los estos pasos:
\begin{enumerate}
	\item Acceder a \gls{liferay}.
	\item Registrarse en el sistema.
	\item Seleccionar la página donde se quiere añadir.
	\item Pulsar en el botón \textit{Add}. Este aparece en la esquina superior izquierda del portal.
	\item Seleccionar \textit{Applications} y luego el portlet de \gls{kf2}.
	\item Arrastrar el elemento de la lista sobre la posición de la página.
\end{enumerate}


\subsubsection{Configuración de la visualización}

También se debe de configurar el \gls{portlet} encargado de la visualización  para que acceda a las funciones del \gls{sw}. Para ello se debe adaptar el fichero $service.properties$ (código \ref{code:visualpropert}).

\begin{listing}[H]
    \begin{minted}[linenos,
               numbersep=5pt,
               frame=single,
               framesep=2mm]{bash}

host=http://localhost:8080
url=/api/jsonws/KnowledgeFinderWebservice.knowledgefinder/
urlDocuments=get-documents/
urlNodes=get-nodes/
    \end{minted}
    \caption{Configuración para la comunicación \gls{portlet} de visualización con el \gls{sw}}
    \label{code:visualpropert}
\end{listing}


\subsubsection*{Configuración en servidor}
En el código \ref{code:liferaydebian} se muestra un ejemplo de configuración para crear un servicio en una máquina de la familia Debian.
%% no incluir listing, sólo referencia al punto anterior

\subsection{\IfLanguageName{english}{Setup Testing}{Pruebas de implantación}}
% Descripción de las pruebas a realizar después de la instalación del sistema. 
Para comprobar que todo el proceso se ha realizado correctamente, simplemente hay que ejecutar algunas de las funciones de la visualización. Si no aparece ningún elemento gráfico en la \gls{ui} puede significar que ésta no se puede comunicar con el \gls{sw}.

\subsection{\IfLanguageName{english}{Operating and Service-level
Procedures}{Procedimientos de operación y nivel de servicio}}
% Procedimientos necesarios para asegurar el correcto funcionamiento, rendimiento, disponibilidad y seguridad del sistema: back-ups, chequeo de logs, etc. También, es preciso indicar claramente aquellas actuaciones precisas necesarias para el mantenimiento preventivo del sistema y así prevenir posibles fallos en el mismo.
Cada petición de \gls{liferay} queda registrado en un sistema de \gls{logging} propio de \gls{tomcat}. Durante toda la ejecución de la visualización se puede consultar a través de los ficheros de la carpeta [UnzipDirectory]\/liferay-portal-6.2-ce-ga2\/tomcat-7.0.42\/logs\/.\\ 

A nivel de \gls{ui}, algunos navegadores como \gls{chrome} o \gls{chromium} incorporan una cosola donde se muestran los errores en tiempo de ejecución y el \gls{logging} de \gls{js}.



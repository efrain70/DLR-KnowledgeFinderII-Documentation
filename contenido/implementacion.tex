% ------------------------------------------------------------------------------
% Este fichero es parte de la plantilla LaTeX para la realización de Proyectos
% Final de Grado, protegido bajo los términos de la licencia GFDL.
% Para más información, la licencia completa viene incluida en el
% fichero fdl-1.3.tex

% Copyright (C) 2012 SPI-FM. Universidad de Cádiz
% ------------------------------------------------------------------------------


\begin{comment}
Este capítulo trata sobre todos los aspectos relacionados con la implementación del sistema en código, haciendo uso de un determinado entorno tecnológico. 
\end{comment}

En este capitulo se detallarán los elementos relacionados con la implementación del \gls{framework} \gls{kf2} y el entorno tecnológico donde se ha desarrollado.


\section{\IfLanguageName{english}{Development environment}{Entorno de
Construcción}}
\begin{comment}
En esta sección se debe indicar el marco tecnológico utilizado para la construcción del sistema: entorno de desarrollo (IDE), lenguaje de programación, herramientas de ayuda a la construcción y despliegue, control de versiones, repositorio de componentes, integración contínua, etc.
\end{comment}
A continuación vamos a describir las herramientas utilizadas  durante el desarrollo resumidas en la sección \ref{section:dlr}.\\

En entorno de desarrollo empleado fue \gls{eclipse} Kepler SR2 utilizándose \gls{java} 1.7 con \gls{jsdk} y \gls{js} como lenguajes de programación principales para el desarrollo del producto.\\

Para controlar las versiones del producto se utilizó un repositorio \gls{svn} instalado en el \gls{dlr} para los proyectos internos el \gls{scvss}. Este repositorio se utilizó conjuntamente con el gestor de proyectos \gls{mantis} y \gls{jenkins} para la integración continua. Ambas herramientas empleadas por el \gls{scvss} para todos sus proyectos.\\

En el \gls{scvss} se dispone de un propio repositorio de \gls{maven} para la gestión y construcción de proyectos. Por ello, esta herramienta fue esencial para la construcción y distribución del proyecto desarrollado.\\

Como se comentó anteriormente, \gls{kf2} utiliza \gls{liferay} como portal y \gls{solr} para el índice de búsqueda. Por este motivo se utilizó una instancia de este portal (versión 6.2.1 CE GA2) funcionando sobre un servidor \gls{tomcat} 7 y de \gls{solr} 4.8.1 sobre \gls{jetty} en su configuración de ejemplo.\\

Para hacer más cómodo el desarrollo, se empleó el \gls{software} \gls{jrebel} 5.6.1 que ayuda a la aplicación automática de la aplicación de las modificaciones en el código \glslink{despliegue}{desplegado}. También se utilizó el \gls{framework} \gls{wiremock} 1.18 para simular los servicios durante la fase \textit{Extended Static Prototype}(ver apartado \ref{subsubsection:aplicandoextre}).\\

En la tabla \ref{table:pluginseclipse} se listan los \glspl{plugin} usados conjuntamente con \gls{eclipse} para integrar las herramientas anteriormente descritas en el entorno de desarrollo y completar la funcionalidad de este \gls{ide}.

\begin{center}
\begin {table}[H]
\centering
    \begin{tabular}{ | l   c   p{7cm} |}
    \hline
    \textbf{\Gls{plugin}} &  \textbf{Versión} & \textbf{Comentario} \\ \hline
    %%%%%%
     Subclipse & 1.10.5 & \Gls{plugin} para \gls{svn} \\ \hline
     Mylyn Mantis Connector & 3.11.0 & \Gls{plugin} para \gls{mantis} \\ \hline
     Hudson/Jenkins Mylyn Builds Connector & 1.4.0 & \Gls{plugin} para \gls{jenkins} \\ \hline
     \Gls{jrebel} for \gls{eclipse} & 5.6.2 & \Gls{plugin} para \gls{jrebel} \\ \hline 
     \Gls{liferay} \gls{ide} & 2.1.1 GA2 & \Gls{plugin} para \gls{liferay} \\ \hline 
     m2e-liferay & 2.1.1 GA2 & \Gls{plugin} para uso conjunto de \gls{maven} y \gls{liferay} \\ \hline
     
    \end{tabular}
    \caption{\Glspl{plugin} usados en \gls{eclipse} durante el desarrollo.}
    \label{table:pluginseclipse}
    \end{table}
\end{center}

\section{\IfLanguageName{english}{Source Code}{Código Fuente}}
\begin{comment}
Organización del código fuente, describiendo la utilidad de los diferentes ficheros y su distribución en paquetes o directorios. Asimismo, se incluirá algún extracto significativo de código fuente que sea de interés para ilustrar algún algoritmo o funcionalidad específica del sistema.
\end{comment}

Como se ha descrito en apartados anteriores, la aplicación está descompuesta en tres subsistemas. A continuación se va a describir el código de cada uno de ellos y los ficheros  claves para la configuración para el proyecto usado como ejemplo \gls{strada}.

\subsection{Índice Solr}
La importación de la información contenida en los repositorios \gls{svn} esta implementada principalmente en los siguientes paquetes de \gls{java}:
\begin{itemize}
	\item \textbf{de.dlr.xps.server.handler.dataimport.crawler}
    En este paquete se implementa la funcionalidad del subsistema \gls{svn} Crawler
    (ver \ref{subsubsection:n2building})
    \item \textbf{de.dlr.xps.server.handler.dataimport.datapicker}
    En este paquete se implementa la funcionalidad del subsistema \gls{svn} DataPicker
    (ver \ref{subsubsection:n2building})
    \item \textbf{de.dlr.xps.server.handler.dataimport.parser}
    En este paquete se implementa la funcionalidad del subsistema \gls{svn} Parser
    (ver \ref{subsubsection:n2building})
    \item \textbf{de.dlr.xps.server.handler.dataimport.transformer}
    En este paquete se implementa la funcionalidad del subsistema \gls{solr} Transformers
    (ver \ref{subsubsection:n2building})
    \item \textbf{de.dlr.xps.server.handler.dataimport}
    En este paquete se implementa la funcionalidad del subsistema \gls{svn} DataImport
    (ver \ref{subsubsection:n2building})
\end{itemize}

Los archivos de configuración para el proyecto \gls{strada} indican cómo debe de realizarse la importación y el esquema del índice de búsqueda. En el fragmento de código \ref{code:import} es observa el fichero \textit{data-conf.xml} resumido para un origen de datos.

\begin{listing}[H]
\begin{minted}[linenos,
               numbersep=5pt,
               frame=single,
               fontsize=\scriptsize,
               framesep=2mm]{xml}
                
<dataConfig>
 <dataSource name="StradaSVN" type="SVNDataSource" />
 <document>
  <entity name="strada" 
   processor="SVNEntityProcessor"
   transformer="
      RegexTransformer,
      TemplateTransformer,
      de.dlr.xps.server.handler.dataimport.transformer.GenerateIdTransformer,
      de.dlr.xps.server.handler.dataimport.transformer.LStripTransformer,
      de.dlr.xps.server.handler.dataimport.transformer.ExcludeValuesTransformer,
      de.dlr.xps.server.handler.dataimport.transformer.CategoriesIdTransformer,
      de.dlr.xps.server.handler.dataimport.transformer.CategoriesSeparatedTransformer,
      de.dlr.xps.server.handler.dataimport.transformer.DateIncompleteFormatTransformer,
      de.dlr.xps.server.handler.dataimport.transformer.SpatialCoverageTransformer,
      de.dlr.xps.server.handler.dataimport.transformer.DictionaryListTransformer"
   
   query="/home/efrain/Escritorio/dlr_software/svn/local/CS">
   <!-- RegexTransformer -->
   <field column="timePeriodOfDataCollection"
          regex="(.*)\s-\s(.*)" 
          groupNames="startCollection,endCollection"/>
   <!-- RegexTransformer -->
   <field column="temporalCoverage" 
   		  regex="(.*)\s-\s(.*)" 
          groupNames="startCoverage,endCoverage"/>
   <!-- DateIncompleteFormatTransformer -->
   <field column="startTemporalCoverage" 
          sourceColName="startCoverage" 
          parseToDate="start" 
          regex="^(?&lt;year&gt;\d{4})(\D)?(?&lt;month&gt;\d{2})?(\D)?(?&lt;day&gt;\d{2})$"/>
   <field column="endTemporalCoverage" 
          sourceColName="endCoverage" 
          parseToDate="end" 
          regex="^(?&lt;year&gt;\d{4})(\D)?(?&lt;month&gt;\d{2})?(\D)?(?&lt;day&gt;\d{2})?$"/>
   <!-- ExcludeValuesTransformer, CategoriesIdTransformer, CategoriesSeparatedTransformer -->
   <field column="contentCategories" 
         name="categories" 
         exclude="import/cs-exclude-categories.txt" 
         categories="import/cs-categories.json" 
         categories_id_name="categories_id" 
         categories_split_prefix="category_"/>
   
   <!-- LStripTransformer, ExcludeValuesTransformer, SpatialCoverageTransformer -->
   <field column="spatialCoverage" 
          lstrip="import/strip-spatial.txt" 
          exclude="import/exclude-spatial.txt"
          spatialColumns="world,regionsWorld,countries,regions"
          worldFile="import/world.txt" 
          worldRegionsFile="import/worldRegions.txt"/>
   <!-- DictionaryListTransformer -->
   <field column="authors" 
          dictionaryId="author" />
  </entity>
  ...
 </document>
</dataConfig>
	\end{minted}
	\caption{Ejemplo de importación de datos en \gls{kf2} para \gls{strada}}
	\label{code:import}
\end{listing}

Para definir el esquema del índice \gls{solr} se utiliza el fichero \textit{schema.xml}. En el código \ref{code:schema} se oberva como se ha definido éste para \gls{strada}.

\begin{listing}[H]
\begin{minted}[linenos,
               numbersep=5pt,
               frame=single,
               fontsize=\scriptsize,
               framesep=2mm]{xml}
                
<?xml version="1.0" encoding="UTF-8" ?>
<schema name="knowledgefinder" version="1.5">
 <fields>
  <field name="_version_" type="long" indexed="true" stored="true" />
  <field name="id" type="string" indexed="true" stored="true" required="true" multiValued="false" />
  <field name="title" type="text_general" indexed="true" stored="true" multiValued="false" required="true" />
  <field name="description" type="text_general" indexed="false" stored="true" multiValued="false"/>
  <field name="keywords" type="string" indexed="true" stored="true" multiValued="true" />
  <field name="categories" type="string" indexed="false" stored="true" multiValued="true" />
  <field name="filePath" type="string" indexed="false" stored="true" multiValued="false" />
  <field name="institute" type="string" indexed="true" stored="true" multiValued="false" />
  <field name="timePeriodOfDataCollection" type="string" indexed="false" stored="true" multiValued="false" />
  <field name="startTimePeriodOfDataCollection" type="tdate" indexed="true" stored="true" multiValued="false" />
  <field name="endTimePeriodOfDataCollection" type="tdate" indexed="true" stored="true" multiValued="false" />
  <field name="startDateDate" type="tdate" indexed="true" stored="true" multiValued="false" />
  <field name="temporalCoverage" type="string" indexed="false" stored="true" multiValued="false" />
  <field name="startTemporalCoverage" type="tdate" indexed="true" stored="true" multiValued="false" />
  <field name="endTemporalCoverage" type="tdate" indexed="true" stored="true" multiValued="false" />
  <field name="spatialCoverage" type="string" indexed="false" stored="true" multiValued="true" />
  <field name="world" type="string" indexed="true" stored="false" multiValued="true" />
  <field name="regionsWorld" type="string" indexed="true" stored="false" multiValued="true" /> 
  <field name="countries" type="string" indexed="true" stored="false" multiValued="true" />
  <field name="regions" type="string" indexed="true" stored="false" multiValued="true" />
  <field name="size_author" type="int" indexed="false" stored="true" multiValued="false" />
  <field name="____contentcreationdatetime____" type="int" indexed="true" stored="false" multiValued="false" />
  <field name="____contentmodificationdatetime____" type="int" indexed="true" stored="false" multiValued="false" />
  <dynamicField name="author_uid_*" type="string" indexed="false" stored="true" multiValued="false" />
  <dynamicField name="author_firstName_*" type="string" indexed="false" stored="true" multiValued="false" />
  <dynamicField name="author_lastName_*" type="string" indexed="false" stored="true" multiValued="false" />
  <dynamicField name="author_mail_*" type="string" indexed="false" stored="true" multiValued="false" />
  <dynamicField name="author_organization_*" type="string" indexed="false" stored="true" multiValued="false" />
  <field name="full_text_search" type="text_general" indexed="true" stored="false" multiValued="true" />
  <field name="title_sort" type="alphaOnlySort" indexed="true" stored="true" multiValued="false" />

  <copyField source="title" dest="full_text_search" />  
  <copyField source="title" dest="title_sort" />
  <copyField source="description" dest="full_text_search" />
  <copyField source="keywords" dest="full_text_search" />
  <copyField source="categories" dest="full_text_search" />
  <copyField source="world" dest="full_text_search" />
  <copyField source="regionsWorld" dest="full_text_search" />
  <copyField source="countries" dest="full_text_search" />
  <copyField source="regions" dest="full_text_search" />
  <copyField source="author_uid_*" dest="full_text_search" />
  <copyField source="author_firstName_*" dest="full_text_search" />
  <copyField source="author_lastName_*" dest="full_text_search" />
  <copyField source="author_mail_*" dest="full_text_search" />
  <copyField source="author_organization_*" dest="full_text_search" />
  <!-- From Transformer Categories -->
  <dynamicField name="category_*" type="string" indexed="true" stored="true" multiValued="true" />
  <copyField source="category_*" dest="full_text_search" />
  <!-- Not logger required -->
  <field name="categories_id" type="ignored" indexed="false" stored="false" multiValued="true" />
  <dynamicField name="*" type="ignored" indexed="true" stored="true" multiValued="true" />
 </fields>
 <uniqueKey>id</uniqueKey>
 ...
	\end{minted}
	\caption{Ejemplo de esquema del índice \gls{solr} en \gls{kf2} para \gls{strada}}
	\label{code:schema}
\end{listing}

\subsection{Servicio Web}
El \gls{sw} que corre sobre \gls{liferay} está organizado en tres paquetes de \gls{java}:
\begin{itemize}
	\sitem{de.dlr.xps.server.knowledgefinder.webservice.solr}
    En este paquete se encuentra el código para obtener la configuración del \gls{sw}.
    \sitem{de.dlr.xps.server.knowledgefinder.webservice.solr.query}
    Para personalizar y configurar las consultas para cada tipo de rol se han definido en este paquete unas clases en modo de ejemplo para los supuestos roles de usuario.
    \sitem{de.dlr.xps.server.knowledgefinder.webservice.service}
    En este paquete se encuentran las clases que definen la \gls{api} del \gls{sw} y su implementación.
\end{itemize}

Para configurar el \gls{sw} y definir el punto de conexión con el servidor \gls{solr} se tiene el fichero \textit{webservice.properties}. En el código \ref{code:service} se puede observar un ejemplo de configuración para el proyecto \gls{strada}.

\begin{listing}[H]
\begin{minted}[linenos,
               numbersep=5pt,
               frame=single,
               framesep=2mm]{bash}
##
## Solr connection
##
	solr.scheme=http
	solr.host=localhost
	solr.port=8983
	solr.username=username
	solr.password =password
	solr.core=solr/strada

    \end{minted}
	\caption{Ejemplo de configuración del \gls{sw} en \gls{kf2} para \gls{strada}}
	\label{code:service}
\end{listing}

\subsection{Visualización}
El subsistema de visualización se puede dividir en grupos de código con propositos claramente diferenciados:

\begin{itemize}
	\sitem{de.dlr.xps.server.knowledgefinder.frontend.controller}
    En este paquete se encuentra el código encargado de inicializar el \gls{portlet} con la para la página web.
    \sitem{de.dlr.xps.server.knowledgefinder.frontend.menu}
    A través de un fichero de configuración, genera el el menú para incorporarlo a la visualización.
    \sitem{Directorio styles-sass}
    En este directorio se encuentra el código en \gls{sass} para definir los estilos de la aplicación.
    \sitem{Directorio js}
    Todo el código \gls{js} creado para la visualización se encuentra alojado en este directorio junto con las librerías empleadas.
\end{itemize}

\subsubsection{Menú}

Para configurar el menú se utiliza el fichero en formato \gls{json} \textit{menuConfiguration.json}. La tabla \ref{table:formatomenu} muestra el formato de los elementos de configuración que lo componen. En el código \ref{code:categorias} se muestra la configuración para \gls{strada}.

\begin{center}
\begin {table}[H]
	\centering
    \begin{tabular}{ | l l |}
    \hline
    \textbf{Propiedad} & \textbf{Comentario} \\ \hline
    %%%%%%
    id & Valor identificativo del elemento  \\ \hline
    name & Nombre mostrado en el menú  \\ \hline
    query & Búsqueda aplicada para el elemento principal \\ \hline
    cssClass & Clase de CSS para el grupo \\ \hline
    subItemsFacet & Consulta facet para obtener los subelementos \\ \hline
    collapsed & Indica si se debe mostrar el grupo plegado \\ \hline
    scrollable & Indica se utiliza un \gls{scroll} para el grupo \\ \hline
    subItems & Lista de subelementos de configuración \\ \hline
    children\_filter & Indica si los subelementos deben ignorarse \\ \hline
    \end{tabular}
    \caption{Formato de los elementos de configuración para el menú}
    \label{table:formatomenu}
  \end{table}
\end{center}


\begin{listing}[H]
\begin{minted}[linenos,
               numbersep=5pt,
               frame=single,
               fontsize=\scriptsize,
               framesep=2mm]{json}
{
  "menuItems": [
    {
      "id": "CAT",
      "name": "Categories",
      "cssClass": "CAT-filter",
      "collapsed": false,
      "scrollable": false,
      "subItems": [
        {
          "id": "PT",
          "name": "Passenger Transport",
          "subItemsFacet": "category_Passenger Transport",
          "cssClass": "CAT-PT-filter",
          "collapsed": true,
          "scrollable": true,
          "children_filter": true
        },
...
      ]
    },
    {
      "id": "SC",
      "name": "Spatial Coverage",
      "cssClass": "SC-filter",
      "collapsed": false,
      "scrollable": false,
      "subItems": [
        {
          "id": "SCW",
          "name": "World",
          "subItemsFacet": "world",
          "cssClass": "SC-SCW-filter",
          "collapsed": true,
          "scrollable": true,
          "children_filter": false
        },
        {
          "id": "SCWR",
          "name": "Word regional level",
          "subItemsFacet": "regionsWorld",
          "cssClass": "SC-SCWR-filter",
          "collapsed": true,
          "scrollable": true,
          "children_filter": true
        },
...
      ]
    },
    {
      "id": "KW",
      "name": "Keywords",
      "cssClass": "red",
      "query": "keywords:[\"\" TO *]",
      "cssClass": "KW-filter",
      "subItemsFacet": "keywords",
      "collapsed": true,
      "scrollable": true,
      "subItems": []
    }
  ]
}
    \end{minted}
	\caption{Ejemplo de configuración del menú en \gls{kf2} para \gls{strada}}
	\label{code:categorias}
\end{listing}

\subsubsection{\gls{js}} 
En la tabla \ref{table:javascriptplug} se muestra las herramientas externas usadas para este proyecto de \gls{js}.

\begin{center}
\begin {table}[H]
	\centering
    \begin{tabular}{ | l l |}
    \hline
    \textbf{Herramienta} & \textbf{Comentario} \\ \hline
    %%%%%%
    \gls{d3} & Herramienta principal de la visualización  \\ \hline
    \gls{bootstrap} & Código para componentes visuales \\ \hline
    jQuery & Herramienta requerida por otras \\ \hline
    Queue.js & Ejecución en paralelo de código \gls{js} \\ \hline
    URI.js & Herramienta para el manejo de \glspl{url} \\ \hline
    History.js & Manipulación del historial para \gls{html5} \\ \hline
    mCustomScrollbar.js & \Gls{scroll} en menú \\ \hline
    \end{tabular}
    \caption{Herramientas de \gls{js} usadas}
    \label{table:javascriptplug}
  \end{table}
\end{center}


\subsubsection{Estilos \gls{css}}

Todos los estilos de la aplicación se han personalizado basándose en la librería \gls{bootstrap} y se han escrito usando \gls{sass}. Para la compilación y aplicación de todos los ficheros fuentes de \gls{sass} se ha usado el \gls{software} \gls{compass}.\\

Para configurar los colores de los elementos gráficos de visualización de la \gls{ui} se ha definido en un único fichero de  \gls{sass}, \textit{\_filter\_colors.scss}, la paleta de colores a usar. En los códigos \ref{code:colorssass} y \ref{code:colorssass2} se muestran cómo es para \gls{strada}.

\begin{listing}[H]
\begin{minted}[linenos,
               numbersep=5pt,
               frame=single,
               fontsize=\scriptsize,
               framesep=2mm]{sass}
               
/* Germany colors
$CAT-color: lighten(#000000, 50%);
$SC-color: #CD0000;
$KW-color: #FFCC00
*/

/* DLR colors */
$CAT-color: rgb(75, 128, 136);
$SC-color: rgb(73, 107, 169);
$KW-color: rgb(126, 75, 86);

$filterColors: (
   CAT-filter: $CAT-color,
   CAT-PT-filter: $CAT-color,
   CAT-T-filter: $CAT-color, 
   CAT-CT-filter: $CAT-color,
   CAT-EI-filter: $CAT-color,
   CAT-VT-filter: $CAT-color,
   CAT-IN-filter: $CAT-color,
   CAT-FDG-filter: $CAT-color,
   CAT-TMP-filter: $CAT-color, 
   
   SC-filter: $SC-color,
   SC-SCW-filter: $SC-color,
   SC-SCWR-filter: $SC-color, 
   SC-SCC-filter: $SC-color,
   SC-SCR-filter: $SC-color,
   
   KW-filter: $KW-color,
);

    \end{minted}
	\caption{Ejemplo de paleta de colores en \gls{kf2} para \gls{strada}}
	\label{code:colorssass}
\end{listing}

\begin{listing}[H]
\begin{minted}[linenos,
               numbersep=5pt,
               frame=single,
               fontsize=\scriptsize,
               framesep=2mm]{sass}
               
@each $filter, $color in $filterColors {
 .#{$filter} {
  /* Graph circles */
  &.gnode {
   > circle {
    fill: $color;
   }
  }
  /* Current selection */
  &.sel-filter {
   color: $color;
   border-color: $color;
  }
  /* Navigation table, selected */
  &.selected{
   .header-filter {
    background-color: rgba($color, 0.25);
   }
  }
 }
}
.CAT-filter {
 .header-group {
  color: white;
  background-color: map-get($filterColors, CAT-filter);
 }
 .header-filter {
  background-color: rgba(map-get($filterColors, CAT-filter), .1);
  
  > a {
   color: map-get($filterColors, CAT-filter);
  }
 }
 .header-subfilter { 
  .name {
   color: map-get($filterColors, CAT-filter);
  }
 }
}
/* Example, how to change color of a subfilter */
/*
.CAT-CT-filter {
 .header-filter {
  background-color: rgba(map-get($filterColors, CAT-CT-filter), .1);
  > a {
   color: map-get($filterColors, CAT-CT-filter);
  }
 }
 .header-subfilter { 
  .name {
   color: map-get($filterColors, CAT-CT-filter);
  }
 }
}
*/
...
    \end{minted}
	\caption{Ejemplo de aplicación de la paleta de colores en \gls{kf2} para \gls{strada}}
	\label{code:colorssass2}
\end{listing}


% \section{\IfLanguageName{english}{}{Scripts de Base de datos}}
% Organización del código fuente, describiendo la utilidad de los diferentes
% ficheros y su distribución en paquetes o directorios. Asimismo, se incluirá el
% script de algún disparador o un procedimiento almacenado, que sea de interés para ilustrar algún aspecto concreto de la gestión de la base de datos.
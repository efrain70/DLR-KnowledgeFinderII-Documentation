\section{El Centro Aeroespacial Alemán (DLR)}
\label{section:dlr}

\begin{wrapfigure}{r}{0.3\textwidth}
  \begin{center}
    \includegraphics[width=0.29\textwidth]{dlr-logo.jpg}
  \end{center}
  \caption{Logotipo \gls{dlr}}
\end{wrapfigure}

El \gls{dlr} es una de las instituciones públicas más importantes dedicadas a la investigación en la República Federal Alemana.\\ 

Sus oficinas centrales se encuentran en la ciudad de Colonia, más de 15 delegaciones, con 32 institutos, repartidas por el territorio nacional alemán y 5 situadas a lo largo del mundo \cite{dlrort}. Actualmente cuenta con más de 7.700 trabajadores y un presupuesto anual dependiente del \gls{bmwi}- de 157 millones de euros para el año 2014 \cite{haushalt2014}.\\


En instituciones orientadas a la investigación, como es el \gls{dlr}, la gestión del conocimiento tiene un significado muy importante. Dado el cambio de personal constante involucrado en las distintas fuentes de conocimiento, reside el peligro de pérdida del mismo. \\

La importancia de la gestión del conocimiento es debido a que éste concreta las actividades de una entidad. Estas actividades ``tienen como meta una mejora de la gestión específica de la organización tanto para el conocimiento interno como externo'' \cite{cissek}. Por lo tanto, el objetivo no debería de ser sólo el almacenamiento sino también la recuperación y reutilización del conocimiento obtenido en situaciones anteriores \cite{dengel}.\\


\subsection{``Transporte y la Movilidad'' en el DLR}

Junto con materias relacionadas con la aeronáutica, el espacio, la energía y la seguridad, el transporte es uno de sus temas de estudio, destacando la evolución de éste como unos de los puntos de trabajo más importantes. Tanto es así que 26 institutos del \gls{dlr} contribuyen en su competencia específica a la mejora del ámbito del transporte. La mayoría de ellos generan gran cantidad de conjuntos de datos estadísticos y bases de datos por lo que la gestión del conocimiento juega un papel clave en puntos como el almacenaje, descripción y su (re)utilización. En el campo de investigación sobre el trasporte y sus datos existen actualmente dos portales en el \gls{dlr}; el ``\gls{monitor}'' del \gls{fw}, implementación del \gls{framework}  \gls{kf}, que aporta datos sobre el transporte aéreo y el portal ``\cs'' del \gls{vf} contribuye con los datos sobre la investigación del transporte.


\subsubsection{\fw{}}
El foco de la investigación de este instituto es examinar cómo los dos sistemas, el tráfico aéreo y el sistema de aeropuertos, evolucionan con paso del tiempo bajo ciertas condiciones y cómo conseguir un estado deseado para cada uno. Para ello, el instituto realiza las siguientes tareas:

\begin{itemize}
  \item Analizar la situación y el desarrollo actual.
  \item Examinar la posible evolución futura de ambos sistemas, por ejemplo, a través de estudios de simulaciones.
  \item La construcción de herramientas de software para ayudar a evaluar las condiciones actuales y evitar situaciones indeseables.
  \item Desarrollar métodos para gestionar los aeropuertos de forma eficiente.
  \item Observar los efectos de las medidas aplicadas.
\end{itemize}

Por lo tanto, el objetivo de la investigación del transporte aéreo es el desarrollo de estrategias y medidas destinadas a introducir cambios en la infraestructura o las nuevas regulaciones para el sistema de transporte aéreo en su conjunto a largo plazo.\\

Tiene como objetivo la investigación del sistema aeroportuario en el desarrollo de medidas para gestionar los aeropuertos de manera eficiente. A través del estudio del movimiento de pasajeros en los procesos de la parte pública y los procesos de parte aeronáutica se intenta fomentar un transporte intermodal eficiente \cite{fwHome}.

%Tiene como objetivo la investigación del sistema aeroportuario en el desarrollo de medidas para la gestión en los aeropuertos de manera eficiente el movimiento de pasajeros en los procesos de la parte pública y la parte aeronáutica fomentando de este modo un transporte intermodal eficiente \cite{fwHome}.

\subsubsection{\vf{}}
Este instituto es el principal proveedor oficial para los hogares alemanes de encuestas y estadísticas relacionadas con el transporte. Aporta datos de sobre el trasporte público y encuestas de movilidad.\\ 

Sus investigaciones se centran en los avances y perspectivas del transporte de pasajeros y comercial para conseguir en el futuro un sistema de transporte moderno, eficiente y sostenible para las personas y el medio ambiente.\\ 

Los campos de investigación del instituto son: 
\begin{itemize}
	\item Estudios de patrones de movilidad sobre viajes domésticos y de negocios de personas.
    \item Modelos para representar y prever la demanda de trasporte regional de pasajeros y comercial.
    \item Evaluación de las tecnologías y medidas en cuanto su posible efectividad.
    \item La aceptación y uso efectivo de la red eléctrica para el trasporte de pasajeros y comercial.
    \item Las interacciones entre la información y la comunicación (TIC) con la movilidad.
\end{itemize}

Este instituto está conectado en red a través de la cooperación en investigación y proyectos a nivel nacional e internacional. Colaborando estrechamente con la docencia e investigación universitaria y la enseñanza en centros de educación superiores y de investigación \cite{vfHome}.